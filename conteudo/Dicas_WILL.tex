% Revisão do texto

% Feedback geral:

% Não esquecer do Resumo/Abstract!

% Evitar frases longas, usar mais ponto

% Evitar parágrafos de uma frase só que podem se juntar a outros
% Dar nomes descritivos aos elementos do jogo ao invés de, por ex., “torre vermelha”. Algo do tipo “torre robusta” ou “asteróide rápido”. Ou usar nomes mais temáticos mesmo, tipo “tanque de patrulha”
.
% Procurem alguma ferramenta de correção ortográfica e revisem o texto antes de entregar

% Comentem a linha dos apêndices se não forem usar

% A quantidade de referências já está ok, mas deixei algumas outras caso consigam incluir

% O índice remissivo em geral é para palavras-chave do trabalho em si, não sei se o que está lá é só um exemplo padrão. Se for, melhor tirar.

% Capítulo 1

% [p. 1] Juntar 1º e 2º parágrafo em um só:X

% [p. 1 / § 2] Antes da última, mencionar que a área de estudos de IA aplicada a jogos existe (citar alguns trabalhos de exemplo) para não ter uma mudança abrupta de assunto.

% [p. 1 / § 3] Citações extras sugeridas:
% Game design: The Art of Game Design por Jesse Schell
% Computação: Game Engine Architecture por Jason Gregory

% [p. 1 / § 5] Ao mencionar jogos, é bom citar a empresa desenvolvedora como se fossem autores, por exemplo: Spelunky (Yu e Hull, 2008): X

% Entre a seção 1.1 e 1.2 era bom dar uma introduzida bem conceitual a algoritmos genéticos. Algo tipo “Existem várias abordagens de IA para produzir essa curva de dificuldade, e algoritmos genéticos chamam atenção por A + B”.

% Capítulo 2

% [p. 3 / § 1] “restrição o escopo” → “restrição do escopo” 

% [p. 3 / § 2] Na primeira frase, acho que fica mais claro dizer “grupos com quantidades finitas de inimigos atacam o jogador”, já que é a divisão dos inimigos em grupos de ataque que caracteriza a mecânica de ondas

% Figuras 2.1 a 2.3
% citar desenvolvedores
% mencionar figuras no texto

% Seção 2.3: citar para Spacewar, Space Invaders e Asteroids

% [p. 5 / § 2] Adicionar notas de rodapé com URL para sites cada tecnologia, com data do último acesso

% Capítulo 3

% Nota de rodapé 1: mover para o texto e usar notas de rodapé para citar URLs dos sites de cada engine, com data do último acesso
% [p. 7 / § 1] frase de quatro linhas - quebrar em frases mais curtas ou, no caso, montar uma lista de itens. Juntar parágrafo com o seguinte.
% [p. 7 / § 2] evitar tantos termos em inglês, pricipalmente quando eles possuem equivalentes comuns em português:
% open source → código aberto
% license → licença
% game engine → motor de jogos (já que vocês mesmos usaram esse termo no parágrafo anterior)

% Figuras 3.1 e 3.2: citar fonte da imagem

% [p. 7 / § 3] escolher ou inglês ou português para se referir aos elementos da Godot e usar de forma consistente (o que ficar em parênteses deve ser só para esclarecer e não o padrão que vocês vão usar). Ajustar no resto da monografia se necessário

% [p. 7 / § 3] sugestão de citação: página de documentação da godot

% [p. 8 / § 2] “todo nó” → “todo nó de colisão/física”

% [p. 8 / § 3] juntar com o parágrafo anterior

% [p. 8 / § 3] “programação uma reação” → “programação de uma reação”

% [p. 9 / § 4] “figura” -> “Figura”

% Figura 3.3: é de vocês? Se for, é legal explicar que ela é do jogo que vocês fizeram. Se não for, citar fonte.

% Figura 3.4: se for um exemplo do jogo de vocês, é legal dizer isso. Faltou um ponto final na legenda.

% [p. 9 / § 4] typos
% “como um objeto filhos” -> “como objetos filhos”

% “a fase geram”
% [p. 10 / § 1] juntar com parágrafo anterior

% Capítulo de conclusão
